\documentclass{article}
\usepackage[pdftex]{graphicx}
\usepackage{hyperref}
\hypersetup{
    colorlinks=true,
    linkcolor=blue,
    filecolor=magenta,      
    urlcolor=blue,
}
\usepackage[document]{ragged2e}
\usepackage{mathtools}
\usepackage{Sweave}
\begin{document}
\Sconcordance{concordance:03-methods.tex:03-methods.Rnw:%
1 11 1 1 0 10 1}


{\large\textbf{Methods}} \\\

In order to properly consult our client, we build score function and apply it to each school, recommending the top institutions to our client. To create the score function we use ordinary least squares regression (OLS) and cross validation. \\\

OLS is an approach to predicting a quantitative response $Y$ based on a multiple predictor variables $X_1$ through $X_p$, where $Y$ and $X_1$ through $X_p$ are vectors and each value in each $X_{ij}$ ($x_{ij}$) has a corresponding value in $Y$ ($y_i$). The model assumes that the relationship between every $X_i$ and $Y$ is linear and that each $X_j$ isn't correlated with any other $X_j$. OLS can be written as $Y \approx \beta_0 + \beta_1X_1 + \beta_2X_2 + ... + \beta_pX_p + \epsilon$, where $\beta_0$ is the intercept and $\beta_1$ through $\beta_p$ are the slopes of their corresponding predictor variable $X_j$. The beta values are all constants, unknowns, and together are the model coefficients. The interpretation of $\beta_0$ is the expected mean value of $Y$ without a predictor variable and the interpretation of $\beta_1$ through $\beta_p$ is the change in $Y$ for a unit increase in the beta's corresponding $X_j$. Although the betas are unknown, we can estimate them using the OLS model: solving for the intercept and slopes that produce the plane closest to each point ($x_{ij}$,$y_i$) in each $X_j$,$Y$, which is minimizing the residual sum of squares ($RSS$). Once we have an estimate for the betas, we can compute using OLS to determine the strength of the relationship between each $X_j$ and $Y$, if the relationship is statistically significant, and asses how accurately the model predicts the relationship. \\\

Cross-validation is a model validation technique used to assess how the results of an analysis will generalize to an independent data set. To cross-validate a model, we fit its required estimated parameters using k parts of a training dataset and then test the k created models on an unseen test dataset. We then chose the model that minimizes the predictive error. If the model performs well using the test dataset (meaning it has a low predictive error), we are more confident that this model is accurate (since it produces good prediction of data not used to fit the model). For our purposes, we will cross validate the resulting model from our OLS regression and choose the $\beta$ vector that has the minimal cross-validated mean squared error.
\end{document}
